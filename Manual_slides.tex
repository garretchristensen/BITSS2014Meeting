% !TEX TS-program = pdflatex
% !TEX encoding = UTF-8 Unicode

% This file is a template using the "beamer" package to create slides for a talk or presentation
% - Talk at a conference/colloquium.
% - Talk length is about 20min.
% - Style is ornate.

% MODIFIED by Jonathan Kew, 2008-07-06
% The header comments and encoding in this file were modified for inclusion with TeXworks.
% The content is otherwise unchanged from the original distributed with the beamer package.

\documentclass{beamer}
%\usepackage{natbib}

% Copyright 2004 by Till Tantau <tantau@users.sourceforge.net>.
%
% In principle, this file can be redistributed and/or modified under
% the terms of the GNU Public License, version 2.
%
% However, this file is supposed to be a template to be modified
% for your own needs. For this reason, if you use this file as a
% template and not specifically distribute it as part of a another
% package/program, I grant the extra permission to freely copy and
% modify this file as you see fit and even to delete this copyright
% notice. 


\mode<presentation>
{
  \usetheme{Warsaw}
  % or ...

  \setbeamercovered{transparent}
  % or whatever (possibly just delete it)
}


\usepackage[english]{babel}
% or whatever

\usepackage[utf8]{inputenc}
% or whatever

\usepackage{times}
\usepackage[T1]{fontenc}
% Or whatever. Note that the encoding and the font should match. If T1
% does not look nice, try deleting the line with the fontenc.


\title[Manual of Best Practices in Transparent Research] % (optional, use only with long paper titles)
{Manual of Best Practices in Transparent Social Sciences Research}

\subtitle
{}

\author[Author, Another] % (optional, use only with lots of authors)
{Garret~Christensen\inst{1} \and Courtney~Soderberg\inst{2}}
% - Give the names in the same order as the appear in the paper.
% - Use the \inst{?} command only if the authors have different
%   affiliation.

\institute[Universities of Somewhere and Elsewhere] % (optional, but mostly needed)
{
  \inst{1}%
  Berkeley Initiative for Transparency in the Social Sciences\\
  UC Berkeley
  \and
  \inst{2}%
  Center for Open Science\\
  }
% - Use the \inst command only if there are several affiliations.
% - Keep it simple, no one is interested in your street address.

\date[BITSS2014] % (optional, should be abbreviation of conference name)
{BITSS Annual Meeting, 2014}
% - Either use conference name or its abbreviation.
% - Not really informative to the audience, more for people (including
%   yourself) who are reading the slides online

\subject{Research Transparency}
% This is only inserted into the PDF information catalog. Can be left
% out. 



% If you have a file called "university-logo-filename.xxx", where xxx
% is a graphic format that can be processed by latex or pdflatex,
% resp., then you can add a logo as follows:

% \pgfdeclareimage[height=0.5cm]{university-logo}{university-logo-filename}
% \logo{\pgfuseimage{university-logo}}



% Delete this, if you do not want the table of contents to pop up at
% the beginning of each subsection:
%\AtBeginSubsection[]
%{
%  \begin{frame}<beamer>{Outline}
%    \tableofcontents[currentsection,currentsubsection]
%  \end{frame}
%}


% If you wish to uncover everything in a step-wise fashion, uncomment
% the following command: 

\beamerdefaultoverlayspecification{<+->}


\begin{document}

\begin{frame}
  \titlepage
\end{frame}

\begin{frame}{Outline}
  \tableofcontents
  % You might wish to add the option [pausesections]
\end{frame}


% Structuring a talk is a difficult task and the following structure
% may not be suitable. Here are some rules that apply for this
% solution: 

% - Exactly two or three sections (other than the summary).
% - At *most* three subsections per section.
% - Talk about 30s to 2min per frame. So there should be between about
%   15 and 30 frames, all told.

% - A conference audience is likely to know very little of what you
%   are going to talk about. So *simplify*!
% - In a 20min talk, getting the main ideas across is hard
%   enough. Leave out details, even if it means being less precise than
%   you think necessary.
% - If you omit details that are vital to the proof/implementation,
%   just say so once. Everybody will be happy with that.

\section{Motivation}

\subsection{Publication Bias}
\begin{frame}{Publication Bias}%{Subtitles are optional.}
  % - A title should summarize the slide in an understandable fashion
  %   for anyone how does not follow everything on the slide itself.
  \begin{itemize}
  \item
   The distibution of published p-values jumps around .05 (Brodeur et al 2013).
  \item
  There is a higher fraction of rejected hypothesis tests in the social sciences (Fanelli 2010).
  \item
  	Published null results are disappearing over time, in all disciplines (Fanelli 2011). 
  \item
    This is very unlikely to represent the true state of the universe.
  \item
  	Data on the complete set of experiments run shows strong results are 40pp more likely to be published, and 60pp more likely to be written up. The file drawer problem is massive. (Franco, Malhotra, Simonovits 2014---see tomorrow)
  \end{itemize}
\end{frame}

\begin {frame}{Publication Bias}
If we only write up/publish significant results, and we have no record of all the insignificant results, we have no way to tell if our `significant' results are real, or if they're the 5\% we should expect due to noise.
\end{frame}

\subsection{P-Hacking}
\begin{frame}{P-Hacking}
\begin{itemize}
\item
Not something only evil people do. It's subconcious.
\item
Also called fishing, researcher degrees of freedom, or data-mining.
\item
Definition: flexibility in data analysis allows portrayal of \textit{anything} as below an arbitrary p-value threshhold; significance loses its meaning.
\end{itemize}
\end{frame}

\begin{frame}
\end{frame}

\section{Solutions}
\subsection{Registration}
\begin{frame}{Definition and Origins}
\end{frame}
\begin{frame}{Contents}
\end{frame}
\subsection{Pre-Analysis Plan}
\begin{frame}{Contents}
\end{frame}

\section{Implementation}
\begin{frame}{AEA Trial Registration}
\end{frame} 
\begin{frame}{EGAP Registration}
\end{frame}
\begin{frame}{3ie Registration}
\end{frame}
\begin{frame}{Open Science Framework}
\end{frame}

\section{Conclusion}
\begin{frame}{Conclusion}
\end{frame}



\end{document}


